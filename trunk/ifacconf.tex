\documentclass{ifacconf}

\usepackage{natbib}            % you should have natbib.sty
\usepackage{graphicx}          % Include this line if your 
                               % document contains figures,
%\usepackage[dvips]{epsfig}    % or this line, depending on which
                               % you prefer.													
% predefined environments
%\begin{thm} ... \end{thm}		% Theorem
%\begin{lem} ... \end{lem}		% Lemma
%\begin{claim} ... \end{claim}	% Claim
%\begin{conj} ... \end{conj}	% Conjecture
%\begin{cor} ... \end{cor}		% Corollary
%\begin{fact} ... \end{fact}	% Fact
%\begin{hypo} ... \end{hypo}	% Hypothesis
%\begin{prop} ... \end{prop}	% Proposition
%\begin{crit} ... \end{crit}	% Criterion

\begin{document}

\begin{frontmatter}

\title{Style for IFAC Conferences \& Symposia: Use Title Case for Paper Title\thanksref{footnoteinfo}} % Title, preferably not more than 10 words.

\thanks[footnoteinfo]{This work was supported in part by the National Technological Agency. (sponsor and financial support acknowledgment goes here). Paper titles should be written in uppercase and lowercase letters, not all uppercase.}

\author[First]{First A. Author} 


\address[First]{National Institute of Standards and Technology, Boulder, CO 80305 USA (Tel: 303-555-5555; e-mail: author@ boulder.nist.gov).}                                              


          
%\begin{keyword}                           % Five to ten keywords,  
%Cicero; Catiline; orations.               % chosen from the IFAC 
%\end{keyword}                             % keyword list or with the 
                                          % help of the Automatica 
                                          % keyword wizard


\begin{abstract}                          % Abstract of not more than 250 words.
These instructions give you guidelines for preparing papers for IFAC conferences. Use this document as a template to compose your paper if you are using Microsoft Word 6.0 or later. Otherwise, use this document as an instruction set. Please use this document as a �template� to prepare your manuscript. For submission guidelines, follow instructions on paper submission system as well as the Conference website. There is a very small blank line immediately above the abstract, do not delete it; it sets the footnote at the bottom of this column.
\end{abstract}

\end{frontmatter}

\section{Introduction}

%Prosumer and BRP
%Domain in context
%aggregaion
%Reduce number of messages recieved by the BRP
%Inroduce intermediate node called an aggregator node
%report of what happens in the rest of the document

\section{related work}

%aggregator
%aggregation problem
%our artikel



\section{Aggregator Node}
%Generalization of the aggregator problem
%strategy
%meassures for the strategy

%Our strategy
%why it is important to balance

\section{Implementation}%low
%webservice
%node reuse



\section{Experimental evaluation}
The data used in this project is from the MeRegio project~\cite{}. 
We only use consumption data in this experiment because we deem that it is not necessary generate production flex-offers to achieve realistic flex-offers.  
All the tests are run in accelerated time.
%accelerated time

We test five strategies that only vary on the set of aggregates, \begin{math} AGG_s \end{math}. The test is conducted with 1, 2, 4, 8, and 16 aggregators. 
To test if the aggregator node reduces the amount of flex-offers received by the BRP we run the experiment with an average of 100, 1000, and 10,000 flex-offers send per time unit. 
%data set

%results

\section{Conclusion and future work}




\section{Conclusion}

A conclusion section is not required. Although a conclusion may review the main points of the paper, do not replicate the abstract as the conclusion. A conclusion might elaborate on the importance of the work or suggest applications and extensions. 

\begin{ack}                               % Place acknowledgements
Partially supported by the Roman Senate.  % here.
\end{ack}

%\bibliographystyle{alpha}        % Include this if you use bibtex 
%\bibliography{autosam}           % and a bib file to produce the 
%\bibliography{autosam}
                                 % bibliography (preferred). The
                                 % correct style is generated by
                                 % Elsevier at the time of printing.

\begin{thebibliography}{xx}

\bibitem[Able(1956)]{Abl:56}
B.C. Able.
\newblock Nucleic acid content of microscope.
\newblock \emph{Nature}, 135:\penalty0 7--9, 1956.

\bibitem[Able et~al.(1954)Able, Tagg, and Rush]{AbTaRu:54}
B.C. Able, R.A. Tagg, and M.~Rush.
\newblock Enzyme-catalyzed cellular transanimations.
\newblock In A.F. Round, editor, \emph{Advances in Enzymology}, volume~2, pages
  125--247. Academic Press, New York, 3rd edition, 1954.

\bibitem[Keohane(1958)]{Keo:58}
R.~Keohane.
\newblock \emph{Power and Interdependence: World Politics in Transitions}.
\newblock Little, Brown \& Co., Boston, 1958.

\bibitem[Powers(1985)]{Pow:85}
T.~Powers.
\newblock Is there a way out?
\newblock \emph{Harpers}, pages 35--47, June 1985.

\bibitem[Soukhanov(1992)]{Heritage:92}
A.~H. Soukhanov, editor.
\newblock \emph{{The American Heritage. Dictionary of the American Language}}.
\newblock Houghton Mifflin Company, 1992.

\end{thebibliography}





\appendix

\end{document}